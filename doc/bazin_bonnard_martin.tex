\documentclass[paper=a4, fontsize=11pt]{article}

\usepackage[margin=0.9in]{geometry}
\usepackage[frenchb.ldf]{babel}
\usepackage[utf8]{inputenc}


% Debut du rapport
\begin{document}

\title{Introduction au TAL\\Compte-rendu de projet}
\author{BAZIN Mathias - BONNARD Nathan - MARTIN Brian}
\date{01/05/2018}
\maketitle

\vspace{1.0cm}

\section{Introduction}

L'objectif de ce projet est de développer un chatbot extrêmement sympathique, une sorte de confident, de meilleur ami avec qui on pourrait discuter de tout et de rien, partager nos souvenirs, raconter nos joies, nos peines, rire, pleurer... En un mot, la personne idéale.
\paragraph{} Ce chatbot, qui parle français et répond au doux nom de Nathanaëlle Poilane, doit être capable d'apprendre à connaître et reconnaitre son interlocuteur. Au fil de la conversation, il retiendra tout ce qui est important : le nom de l'utilisateur, ses amis, sa famille, ses goûts, les activités qu'il aime pratiquer... et ce même si l'on quitte l'application, grâce à un système de sauvegarde.

\section{Calou et Nathanaëlle}

Le programme est organisé en trois modes, chacun plus complexe que le précédent. L'utilisateur peut changer de mode à tout moment, et si le mode actuel n'est pas capable de répondre correctement, il bascule sur le mode inférieur.

\subsection{Mode 1}

Dans ce mode, Nathanaëlle cède la place à Calou, un adorable chien. Comme tous les représentants de son espèce, il n'est pas capable de tenir une conversation hautement intellectuelle, mais peut néanmoins réagir à certains mots, comme son prénom.

\subsection{Mode 2}

Dans ce mode, Nathanaëlle est capable de tenir une conversation basique sur certains sujets : les animaux, la dépression, [...]. Elle est capable d'appliquer certaines règles de la langue française, notamment l'accord masculin/féminin/pluriel. De plus, elle répond aux phrases du style "Je suis X" par "Pourquoi es-tu X ?", et ce à tous les temps de l'indicatif, et même si X est précédé d'un quantifieur. Par exemple, la réponse à "Je serai très fatigué demain." sera "Pourquoi seras-tu fatigué ?".

\subsection{Mode 3}

Le mode 3 est le mode le plus complet. Le chatbot détecte et enregistre des informations relatives à l'utlisa

\section{Contributions des membres}

\end{document}