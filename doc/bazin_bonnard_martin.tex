\documentclass[paper=a4, fontsize=11pt]{article}

\usepackage[margin=0.9in]{geometry}
\usepackage[frenchb.ldf]{babel}
\usepackage[utf8]{inputenc}


% Debut du rapport
\begin{document}

\title{Introduction au TAL\\Compte-rendu de projet}
\author{BAZIN Mathias - BONNARD Nathan - MARTIN Brian}
\date{01/05/2018}
\maketitle

\section{Introduction}

L'objectif de ce projet est de développer un chatbot extrêmement sympathique, une sorte de confident, de meilleur ami avec qui on pourrait discuter de tout et de rien, partager nos souvenirs, raconter nos joies, nos peines, rire, pleurer... En un mot, la personne idéale.\\
Ce chatbot doit être capable d'apprendre à connaître et reconnaitre son interlocuteur. Au fil de la conversation, il retiendra tout ce qui est important : le nom de l'utilisateur, ses amis, sa famille, ses goûts, les activités qu'il aime pratiquer... et ce même si l'on quitte l'application, grâce à un système de sauvegarde.

\section{Calou et Nathanaëlle}

\section{Contributions des membres}

\end{document}